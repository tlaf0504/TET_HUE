% !TEX root = main.tex

\newchapter{Finite Differenzen Methode}{}
\newsection{Theoretische Grundlagen}{}
Das zu lösende Randwertproblem ist wie folg definiert:
\begin{align}
	\label{eq:Randwretproblem}
	-\varepsilon \Delta V &=  \rho  \text{ mit }\\
	V &= V_D \text{ auf } \Gamma_D \nonumber\\
	\pDiff{V}{n} &= V_N \text{ auf } \Gamma_N \nonumber
\end{align}

wobei \begin{align}
	\label{eq:LaplaceV}
	\Delta V = \pDiffDiff{V}{x} + \pDiffDiff{V}{y} + \pDiffDiff{V}{z}
\end{align}

In den gegebenen Beispielen ist $\varepsilon$ konstant und $\neq 0$, $\rho = 0$ und $V_N = 0$. Des Weiteren gilt aufgrung der Ausdehnung des Problems $\pDiff{}{z} = 0$, womit sich das Randwertproblem (\ref{eq:Randwretproblem}) wie folgt vereinfacht:
\begin{align}
	\label{eq:Randwertproblem_simple}
	&\pDiffDiff{V}{x} + \pDiffDiff{V}{y} = 0 \text{ mit }\\
	&V = V_D \text{ auf } \Gamma_D \nonumber\\
	&\pDiff{V}{n} = 0 \text{ auf } \Gamma_N \nonumber
\end{align}

Man nehme nun eine konstante \textit{Schrittweite} $h$ an und führe folgende Notation ein:
\begin{align*}
	V_{i,j} = V(ih,jh),\quad i,j\geq 0
\end{align*}

Die Linearisierung der Differentialgleichung in \myRef{eq:Randwertproblem_simple} liefert somit
\begin{align*}
	\frac{V_{i+1,j} - 2V_{i,j} + V_{i-1,j}}{h^2} + \frac{V_{i,j+1} - 2V_{i,j} + V_{i,j-1}}{h^2} = 0
\end{align*}
bzw. unter Berücksichtigung der Konstanz von $h$
\begin{align}
	\label{eq:lin_eq}
	-4V_{i,j} + V_{i+1,j} + V_{i-1,j} + V_{i,j+1} + V_{i,j-1} = 0.
\end{align}

Das Problemgebiet wird somit in Quadrate der Seitenlänge $h$ diskretisiert wobei die \textit{Knotenpotentiale} an den Gitterknotenpunkten durch Lösen des aus \myRef{eq:lin_eq} entstehenden linearen Gleichungssystems ermittelt werden. Die vorgegebenen Potentiale am dirichletschen Rand $\Gamma_D$ fließen als bekannte Größen in das Gleichungssystem ein. \newline
Ferner sei ohne weitere Herleitung gesagt, dass die Behandlung der \textbf{homogenen} neumannschen Randbedingungen keiner weiteren Aufmerksamkeit bedarf. Hier sei auf die entsprechende Fachliteratur verwiesen. \newline
Die genaue Vorgehensweise zur Lösung des Problems mit der oben genannten Methode wird im folgenden Kapitel anhand eines einfachen Beispiels erläutert.

\newsection{Funktion des FDM-Algorithmus}{}
	
	

     


